%%%%%%%%%%%%%%%%%%%%%%%%%%%%%%%%%%%%%%%%%
% Journal Article
% LaTeX Template
% Version 1.4 (15/5/16)
%
% This template has been downloaded from:
% http://www.LaTeXTemplates.com
%
% Original author:
% Frits Wenneker (http://www.howtotex.com) with extensive modifications by
% Vel (vel@LaTeXTemplates.com)
%
% License:
% CC BY-NC-SA 3.0 (http://creativecommons.org/licenses/by-nc-sa/3.0/)
%
%%%%%%%%%%%%%%%%%%%%%%%%%%%%%%%%%%%%%%%%%

%----------------------------------------------------------------------------------------
%	PACKAGES AND OTHER DOCUMENT CONFIGURATIONS
%----------------------------------------------------------------------------------------

\documentclass[twocolumn,UTF8]{article}

\usepackage{blindtext} % Package to generate dummy text throughout this template

\usepackage[sc]{mathpazo} % Use the Palatino font
\usepackage[T1]{fontenc} % Use 8-bit encoding that has 256 glyphs
\linespread{1.05} % Line spacing - Palatino needs more space between lines
\usepackage{microtype} % Slightly tweak font spacing for aesthetics

\usepackage[english]{babel} % Language hyphenation and typographical rules

\usepackage[hmarginratio=1:1,top=32mm,columnsep=20pt]{geometry} % Document margins
\usepackage[hang, small,labelfont=bf,up,textfont=it,up]{caption} % Custom captions under/above floats in tables or figures
\usepackage{booktabs} % Horizontal rules in tables

\usepackage{lettrine} % The lettrine is the first enlarged letter at the beginning of the text

\usepackage{enumitem} % Customized lists
\setlist[itemize]{noitemsep} % Make itemize lists more compact

\usepackage{abstract} % Allows abstract customization
\renewcommand{\abstractnamefont}{\normalfont\bfseries} % Set the "Abstract" text to bold
\renewcommand{\abstracttextfont}{\normalfont\small\itshape} % Set the abstract itself to small italic text

\usepackage{titlesec} % Allows customization of titles
\renewcommand\thesection{\Roman{section}} % Roman numerals for the sections
\renewcommand\thesubsection{\roman{subsection}} % roman numerals for subsections
\titleformat{\section}[block]{\large\scshape\centering}{\thesection.}{1em}{} % Change the look of the section titles
\titleformat{\subsection}[block]{\large}{\thesubsection.}{1em}{} % Change the look of the section titles

\usepackage{fancyhdr} % Headers and footers
\pagestyle{fancy} % All pages have headers and footers
\fancyhead{} % Blank out the default header
\fancyfoot{} % Blank out the default footer
\fancyhead[C]{ZJUT International College} % Custom header text
\fancyfoot[RO,LE]{\thepage} % Custom footer text

\usepackage{titling} % Customizing the title section

\usepackage{hyperref} % For hyperlinks in the PDF

\usepackage{ctex}

%----------------------------------------------------------------------------------------
%	TITLE SECTION
%----------------------------------------------------------------------------------------

\setlength{\droptitle}{-4\baselineskip} % Move the title up

\pretitle{\begin{center}\Huge\bfseries} % Article title formatting
\posttitle{\end{center}} % Article title closing formatting
\title{The Various Aspects of Outsourcing} % Article title
\author{
\textsc{Xueliang Wu}\\[1ex]
\normalsize Zhejiang University of Technology \\
\normalsize \href{mailto:xlwu25@gmail.com}{xlwu25@gmail.com}
}
\normalsize \date{\today}
\renewcommand{\maketitlehookd}{
\begin{abstract}
\noindent The document mainly talks about the various aspects of outsourcing, inlcude the factors affecting outsourcing, the risks in outsourcing, the benefits associated with an outsourcing effort, the models in outsourcing, personnel issues in outsouring, the costs of outsource and the impact of it.\\
\textbf{ Keywords: Outsourcing, Costs, Benefits, Risks, Influence}
\end{abstract}
}

%----------------------------------------------------------------------------------------

\begin{document}

% Print the title
\maketitle

%----------------------------------------------------------------------------------------
%	ARTICLE CONTENTS
%----------------------------------------------------------------------------------------

\section{Introduction}

\lettrine[nindent=0em,lines=2]{W} ith the development of the world economy, information technology and corporate strategy, outsourcing has been greatly changed, and the strategic position of outsourcing is improving continuously.
So we need to learn, understand, and master outsourcing.\\
What's the outsouring? In brief, outsourcing refers to the completion of the work done within the enterprise by purchasing a service product from third party providers. \\
What kind of knowledge we should know? in this document, I will answer the following questions:
\begin{itemize}
\item What are the determining factors that may lead to a decision to proceed with outsourcing? What factors might lead the manager to not consider outsourcing?
\item What are the risks associated with outsourcing? And what methods can be used to alleviate the risks?
\item What are the benefits associated with an outsourcing effort?
\item What models can be applied for outsourcing project? Evaluate the models pros and cons.
Where do the costs originate in an outsourcing agreement, and what are some examples of the dollar impacts that might be expected?
\item What are the implications to the business organizational structure by using an outsourced IT department as well as the potential personnel issues which may arise?
\end{itemize}

%------------------------------------------------

\section{Methods}

\subsection{The factors affecting outsourcing}
A firm may outsource none, some, or many of these functions. I thought three key factors influence outsourcing decision:\\
\indent 1.Centrality of the functions to the firm's core competency; \\
\indent 2.Risk liability and control; \\
\indent 3.Cost/service tradeoffs in operations; \\

\subsubsection{Centrality of the functions to the firm's core competency}
A firm's core competency is very important for a firm. The decision makers should think deeply, determine if the project which the firm want to outsourcing is the core competency or not, if it is, then it'd better not to outsource. Because core competency is the foundation of a company's survival.

\subsubsection{Risk evaluation and control}
Risk determines whether an outsourcing project is successful or not, if the risk is too high, the failure of the outsourcing project will have a high probability. And if we have a good risk evaluation and control, the probability of success would be nice. So risk evaluation and control play a impotant role in decision.

\subsubsection{Cost/service tradeoffs in operations}
Cost is an important factor in outsourcing. If the cost is too high, it called
\textquotedblleft The loss outweighs the gain\textquotedblright.

%------------------------------------------------

\subsection{Risks of Outsourcing}

\subsubsection{The definition of risk of outsourcing}
The risk is a kind of possibility that the environmental and conditional uncertainty of outsourcing and the impact that stakeholders can not accurately predict or control lead to a discrepancy between the final results and the expectations of the stakeholders.

\subsubsection{The risks associated with outsourcing}
Many people have studied the risk associated with outsourcing at home and abroad. \\
James Brain Quinn, Frederick G. and Hilmer stated the three main risks of outsourcing include:
\begin{itemize}
\item The risk of losing critical capacity or development ability.
\item The risk of losing the interactive ability in stages of research and development, production, marketing.
\item The risk of losing the control of the docking contractors.
\end{itemize}
Robert Klepper and Wendello. Jones summarized the risks of IT outsourcing. They are
\begin{itemize}
\item The primary risk faced by companies is out of control. When something is outsourced, companies can not control things directly so that companies may not get feedback quickly and can not make improvements in time.
\item Uncertainty is the inherent risk faced by companies.
\item Companies will face many potential pitfalls. such as, the misunderstanding of information technology leads to the core role of information technology resources or business being outsourced.
\end{itemize}
Liu Jingjiang, Liu Guihua (2004) thought the risks of IT outsourcing mainly include:
The subcontractor becomes a new competitor; the subcontractor fails to carry out the contract; the subcontractor extracts the profit from the supply chain. \vspace{6pt} \\
Yang Juan, Liu Qiang and Yang Jing (2004) believed that the risks of outsourcing are the risk of losing control, the risk of uncertain distribution of income, the negative impact of outsourcing on employees.\vspace{6pt} \\
Wang Min (2001) thought that the lack of effective communication is one of the biggest risks of outsourcing.\\

From my perspective, the risk of outsourcing divided into two parts. One of them is from the perspective of outsourcing result. In this part, the risks of outsourcing mainly are the contractors can not meet the require of the enterprise for information technology, communication problems between companies and contractors, the cost of outsourcing may exceed the value gained by outsourcing, etc. All this factors may lead to the failure of outsourcing. Another is from the perspective of company itself. In this part, the risks of outsourcing mainly are outsourcing may weaken the ability of R \& D and innovation to lose competitiveness, the contractor may be a potential competitor, less communication with customers, easy to lose customers, etc. All these factors can influence the company development.


\subsubsection{Management can be used to alleviate the risks}
The basic idea is to have controls in place that minimize the negative consequences of a bad outsourcing agreement, known as risk management.
Risk management consists of three closely related actions: Risk identification, Risk analysis, Risk control.
\begin{itemize}
\item \textbf{Risk identification:} Different outsourcing project has different risks, so we should identify the types of this outsourcing project, list the possible risks as much as possible, then arrange the risk list, remove the items that are not possible.
\item \textbf{Risk analysis:} After identifying the risks, the next action is to collect information, then analyze if it possible to reduce the risk.
\item \textbf{Risk control:} There are two categories of controls: preventive and corrective. Preventive controls mitigate a threat from exploiting the vulnerabilities of a project. Corrective controls require addressing the impact of a threat and then establishing controls to preclude any future impacts.
\end{itemize}

%------------------------------------------------

\subsection{The Benefits Associated with An Outsourcing Effort}
There are many benefits associated with an outsourcing effort, such as:
\begin{itemize}
\item \textbf{Improving effciency:} In general, firm is not adept in the aspect that firm want to outsource, so if they not outsource, the effciency will not be high.
\item \textbf{Getting access to speciallized skills:} The subcontractor usually has its own unique technology, Contacting more with the subcontractor, it's equivalent to get access to speciallized skills.
\item \textbf{Saving on manpower and training costs:} Some employees do not have the ability to do the project, so if not outsource, there must have some training costs.
\item \textbf{Improving speed and service:} More people do it, so it can improve speed and service.
\item \textbf{Offload non-core function:} It can help firm specialize in core competency, improve quality of service, and offload non-core competency.
\end{itemize}


%------------------------------------------------
\subsection{The Models in Outsourcing}
To be defined

%------------------------------------------------
\subsection{Personnel Issues in Outsouring}
To be defined

%------------------------------------------------
\subsection{The Costs of Outsource and The Impact of It}
To be defined


%------------------------------------------------

\section{Conclusion}
To be defined



%----------------------------------------------------------------------------------------
%	REFERENCE LIST
%----------------------------------------------------------------------------------------

\begin{thebibliography}{99} % Bibliography - this is intentionally simple in this template

\bibitem[1]{} Kant Rao, Richard R. Young, (1994) Global Supply Chains: Factors Influencing Outsourcing of Logistics
Functions, International Journal of Physical Distribution \& Logistics Management, Vol. 24 Issue: 6, pp.11-19.
\bibitem[2]{} Ralph L. Kliem (1999) Managing the Risks of Outsourcing Agreements, Information Systems Management, 16:3, 91-93.
\bibitem[3]{} 周旭. 服务外包风险的识别与控制 [D]. 西安:西安电子科技大学,2009.

\end{thebibliography}

%----------------------------------------------------------------------------------------

\end{document}
